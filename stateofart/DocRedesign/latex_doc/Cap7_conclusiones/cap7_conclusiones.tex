% traer conclusiones de trabajos futuros que hay en otros capitulos mas arriba

% poner referencia a shows que hicimos, ingenier�a de muestra y noche de fallos farq.

% Comentar los problemas que surgieron, que el software no estaba listo y los agregados que se le hicieron al software a partir de esto.
% Al principio el formato del show estaba distribuido en todos los nodos





%Trabajos a futuro:

% Posicionar y editar quads y objetos3d con el mouse.

% Visualizaci�n del show completa integrando 2d con 3d, o sea, mostrar los quads proyectados sobre la escena 3d y no que solamente se vean en el nodo proyector.

% Un punto posible de extensi�n es la creaci�n de m�s efectos gen�ricos como pueden ser animaci�n de rotaci�n, escala o modificadores de objetos tridimensionales, para dar mas posibilidades al usuario al momento de la creaci�n del espect�culo.

% Extensibilidad a trav�s de plugins, ser�a complejo, habr�a que soportar adem�s un mecanismo de crear interfaces (desde el plugin) para configurar los mismos plugins.
