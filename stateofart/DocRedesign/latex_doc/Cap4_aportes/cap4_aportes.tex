\chapter{Aportes}
A continuación se presentan los resultados de una serie de entrevistas realizadas a \emph{VJs} e ingenieros contactados a partir de nuestra investigación del estado del arte del \emph{video mapping}. Se les consultó sobre sus trabajos actuales, y las técnicas y herramientas que utilizan para su producción.
Fueron introducidos brevemente al proyecto para recibir comentarios y conocer sus expectativas en cuanto a herramientas que puedan asistirlos en la resolución o simplificación de problemas con los que se encuentran actualmente al momento de realizar los espectáculos de \emph{video mapping}.
\footnote{Agradecemos a Marcelo Vidal, Martin Borini, Viktor Vicsek y Kyle McDonald por sus aportes.}

\section{Marcelo Vidal}
Marcelo Vidal\cite{Chindogu} es un \emph{VJ} local que se está posicionando como referente del \emph{video mapping} en nuestro medio y quien ha desarrollado los espectáculos más importantes y de más repercusión local.
Marcelo Vidal comenta que para sus espectáculos trabaja principalmente con modelos en dos dimensiones, por lo que la fase inicial de su trabajo se basa en transformar la fachada o superficie a mapear en un modelo bidimensional. Para este propósito utiliza \emph{AfterEffects}, \emph{Photoshop} y \emph{3D Studio}, %(referenciar)
fijando la posición y orientación del proyector en el lugar desde donde se realizará la proyección, para luego capturar las formas y construir el modelo bidimensional.
Una vez modelada la escena, se realiza todo el trabajo creativo y de diseño visual del espectáculo, para luego volver a la escena, posicionar nuevamente el proyector en el lugar original, y comenzar a realizar los ajustes previos. Entre estos ajustes se encuentra la calibración y posicionamiento del proyector en donde fue resaltada la dificultad existente para mover los proyectores que se utilizan en espectáculos de gran porte, los que pueden llegar a pesar hasta 200 kilogramos y no es posible realizar movimientos milimétricos para ajustes finos. Por esto último es que considera importante que luego de armar la escena siempre será necesario un ajuste o calibración del modelo de software sobre la superficie. Destacó que utiliza exactamente el mismo proyector tanto para la extracción de la geometría como para la posterior ejecución del espectáculo de \emph{video mapping}.
En cuanto al mapeo, fue destacado como muy importante poder tener la posibilidad de modificar en tiempo real el espectáculo. En cuanto al tratamiento de las deformaciones que se producen al proyectar sobre una superficie irregular comentó que maneja diferentes estrategias. Puede tanto utilizar la deformación dada por la superficie junto con deformaciones de la imagen o video a proyectar para lograr un efecto en conjunto, como intentar modificar el video o imagen para minimizar los efectos dados por la superficie irregular. Nos comenta que esta última opción es la menos interesante en el ámbito artístico de los \emph{VJs}. Las herramientas que utiliza para la realización de sus espectáculos son \emph{QuickTime}, \emph{Module8}, \emph{VDMX} y \emph{Resolume PC}.

\section{Martin Borini (\emph{VJ Ailaviu})}
La principal inquietud que nos transmitió Martín Borini\cite{Ailaviu} fue en relación al manejo de información tridimensional durante el proceso de producción de un espectáculo y así poder conocer las deformaciones que se darán en la superficie donde se proyectará. También se muestra preocupado por el problema de la correspondencia entre el modelo tridimensional y la superficie, lo que lo lleva a estar interesado en la posibilidad de la calibración de un modelo de ese estilo.
En sus trabajos realizados sobre fachadas ha tomado fotos a nivel desde el mismo lugar donde se coloca el proyector. Según su experiencia es mejor que trabajar sobre planos o medidas tomadas por él. Otro punto en el cual expresó interés fue en casos en los que se utilizan más de un proyector donde surge el problema a solucionar de las costuras\footnote{glosario con explicación que sigue a este paréntesis de como usamos el termino costura en el contexto del video mapping} o uniones de las imágenes proyectadas por los diferentes proyectores.

\section{Viktor Vicsek}
Viktor Vicsek\cite{Viktorvicsek} es un \emph{VJ} de nivel internacional y se ha establecido contacto con él luego de observar los espectáculos de más relevancia y popularidad referenciados en sitios de internet. Comenta que ha implementado sus propios módulos de software para asistir en la producción. También ha realizado una mejora a la técnica de calibración automática de proyectores de Johnny Lee\footnote{http://johnnylee.net/projects/thesis} utilizando cámaras y visión por computadora en lugar de sensores de luz, la que le ha resultado particularmente útil para el escaneo de grandes superficies o fachadas de edificios. También ha implementado su propia línea de tiempo de videos utilizando \emph{Adobe Air}\footnote{Adobe Air}. Para lo que es la realización de los mapeos y las distorsiones de imágenes y videos utiliza la herramienta de programación visual para mapeo \emph{VVVV}.

Su proceso de producción consta en tomar fotografías y obtener los planos arquitectónicos de la escena en cuestión y la creación de máscaras según se necesite con herramientas de modelado como \emph{3D Studio}. Posteriormente realiza el mapeo de texturas de imagen o video aplicando efectos que utilizan \emph{shaders}\footnote{Shaders} sobre las máscaras mediante la utilización de \emph{VVVV}. Para el caso de la utilización de múltiples proyectores en el que es necesario resolver el problema de costuras, Vicsek aplica \emph{edge blending}\footnote{edge blending} y homografías a una máscara adicional.

\section{Kyle McDonald}%Me deja muchas dudas toda la sección
Kyle McDonald\cite{KyleMcDonald} fue contactado inicialmente en referencia a su implementación de código abierto del mecanismo de \emph{structured light}\footnote{Structured Light} para el escaneo y reconstrucción de objetos tridimensionales, específicamente utilizando \emph{phase-shift scanning} y basándose en la técnica de Zhang\footnote{Zhang's three-phase}. McDonald fue consultado por ciertos problemas específicos %%%se podrían explicar cuales son estos problemas específicos o no poner nada%%%
que se encontraron al utilizar su método de escaneo para nuestro proyecto, nos confirma que tiene como requerimiento que la escena este compuesta por elementos que resulten en una forma continua. Es por este motivo que la cara de un ser humano puede ser correctamente escaneada, pero no objetos simples aislados que formen parte de la misma escena. Este último caso a modo de simulación de lo que podría ser una fachada o superficie irregular en general a ser mapeada. Nos sugirió algunas alternativas para sobreponernos a este problema, basándose en que la técnica de Zhang utiliza valores de profundidad propagados a través de la superficie, por lo tanto si dos superficies son discontinuadas no es posible determinar las profundidades relativas a los dos objetos de la escena. Para ello podríamos utilizar escáneres que utilizan métodos híbridos basado en proyección de múltiples patrones junto con utilización de información espacial, particularmente la consideración de los puntos vecinos en la decodificación\footnote{ref7?}. En este grupo de escáneres encontramos \emph{BYO3D}\footnote{http://mesh.brown.edu/byo3d/} y \emph{David-laserscanner}\footnote{http://www.david-laserscanner.com}.

Como una inquietud general, nos comenta que actualmente está muy interesado en trabajar con sistemas de mapeo en tiempo real. Propone la utilización de sensores infrarrojos para escanear la escena en movimiento, citando ejemplos de actores o bailarines en un teatro.