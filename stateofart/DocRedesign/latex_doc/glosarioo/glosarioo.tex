\Huge
\textbf{Glosario}

\vspace{10 mm}

\normalsize 

\paragraph{VJ}
Un \emph{Video Jockey} es un artista quien genera sesiones visuales mezclando en directo pistas de video con música u otro tipo de elemento.

\paragraph{Motor tridimensional}
Término que hace referencia a una serie de rutinas de programación que permiten el diseño, creación y representación de aplicaciones tridimensionales en tiempo real.
El motor tridimensional acepta comandos de una aplicación y construye imágenes y texto que son dirigidos al hardware gráfico.
 
\paragraph{Nube de puntos}
Colección de vértices que definen la superficie externa de un objeto.

\paragraph{Malla tridimensional}
Conjunto de vértices, aristas y caras en un sistema de coordenadas tridimensional que representan la superficie externa de un objeto.

\paragraph{Visión por computadora}
Rama de la inteligencia artificial que mediante el procesamiento de imágenes por computadora extrae información de las mismas para la toma de decisiones.

\paragraph{Mapa de bits}
Estructura de datos que representa una rejilla rectangular de píxeles que se puede visualizar en un monitor u otro dispositivo de representación.

\paragraph{Algoritmo de Kruskal}
Algoritmo de la teoría de grafos para encontrar un árbol de expansión mínimo en un grafo conexo y ponderado. 
%Es decir, busca un subconjunto de aristas que, formando un árbol, incluyen todos los vértices y donde el valor total de todas las aristas del árbol es el mínimo.%

\paragraph{Eje óptico}
Eje perpendicular al plano imagen que pasa por el centro óptico de la cámara.

\paragraph{GPU}
Procesador suplementario a la unidad de procesamiento central cuya función específica es el procesamiento de gráficos.

\paragraph{Canal alfa}
Porción del píxel en el cual se guarda información de su opacidad.

\paragraph{Píxel}
Acrómino del inglés \emph{picture element} o elemento de imagen, es la unidad básica de color de una imagen digital.

\paragraph{UDP}
Protocolo usado ampliamente para consultas de petición y respuesta de una sola ocasión, del tipo cliente-servidor y en aplicaciones en las que la entrega pronta es más importante que la entrega confiable, como las trasmisiones de voz y video.
