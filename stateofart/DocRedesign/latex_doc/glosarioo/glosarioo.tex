\Huge
\textbf{Glosario}

\vspace{10 mm}

\normalsize 

\paragraph{Video Jockey}
También conocido como VJ es quien genera sesiones visuales mezclando en directo pistas de video con música u otro tipo de elemento.

\paragraph{Motor tridimensional}
Término que hace referencia a una serie de rutinas de programación que permiten el diseño, la creación y la representación de aplicaciones tridimensionales en tiempo real.
El motor tridimensional acepta comandos de una aplicación y construye imágenes y texto que son dirigidos al hardware gráfico.
 
\paragraph{Nube de puntos}
Colección de vértices que definen la superficie externa de un objeto.
A diferencia de las mallas tridimensionales, las nubes de puntos no poseen aristas ni caras.

\paragraph{Malla tridimensional}
Conjunto de vértices en un sistema de coordenadas tridimensional que representan la superficie externa de un objeto.

\paragraph{Visión por computadora}
Rama de la inteligencia artificial y procesamiento de imágenes que estudia el procesamiento de imágenes del mundo real. La visión por computadora típicamente requiere una combinación de procesamiento de imágenes a bajo nivel para mejorar la calidad de la imagen y de reconocimiento de patrones a alto nivel.

\paragraph{Mapa de bits}
Estructura o fichero de datos que representa una rejilla rectangular de píxeles que se puede visualizar en un monitor, papel u otro dispositivo de representación.

\paragraph{Algoritmo de Kruskal}
Algoritmo de la teoría de grafos para encontrar un árbol recubridor mínimo en un grafo conexo y ponderado. Es decir, busca un subconjunto de aristas que, formando un árbol, incluyen todos los vértices y donde el valor total de todas las aristas del árbol es el mínimo.

\paragraph{Eje óptico} 
Eje perpendicular al plano imagen que pasa por el centro de la cámara, también llamado centro óptico\cite{OpticalDesign}.

\paragraph{Graphic Processing Unit (GPU)}
La unidad de procesamiento gráfico es un procesador suplementario a la unidad de procesamiento central cuya función especifica es el procesamiento de gráficos\cite{GPUWork}.

\paragraph{Canal alfa}
Canal en el cual se guarda información de opacidad de cada píxel\cite{3DGraphics}.

\paragraph{Píxel}
Superficie homogénea más pequeña de las que componen una imagen, que se define por su brillo y color\cite{RAE}.

\paragraph{OSC}
\emph{Open Sound Control} es un protocolo para comunicación entre computadoras, sintetizadores de sonido y otros dispositivos multimedia que está optimizado para la tecnología de red moderna\cite{OSCProtocol}.

\paragraph{UDP}
Protocolo de datagrama de usuario, es un protocolo sin conexión, no confiable, para aplicaciones que no necesitan la asignación de secuencia ni el control de flujo del TCP y que desean utilizar los suyos propios. Este protocolo también se usa ampliamente para consultas de petición y respuesta de una sola ocasión, del tipo cliente-servidor y en aplicaciones en las que la entrega pronta es más importante que la entrega precisa, como las trasmisiones de voz y vídeo\cite{Tanenbaum}.