\chapter{Aportes}
\label{chapter:aportes}
A continuación se presentan los resultados de una serie de entrevistas realizadas a \emph{VJs} e ingenieros contactados a partir de la investigación del estado del arte del \emph{video mapping}. Se les consultó sobre sus trabajos realizados y las técnicas y herramientas que utilizaron para su producción.
Fueron introducidos brevemente al proyecto para recibir comentarios y conocer sus expectativas en cuanto a herramientas que puedan asistirlos en la resolución o simplificación de problemas con los que se encuentran actualmente al momento de realizar espectáculos de \emph{video mapping}.

\section{Marcelo Vidal (VJ Chindogu)}
Marcelo Vidal\cite{Chindogu} es un \emph{VJ} local referente del \emph{video mapping} en nuestro medio y quien ha desarrollado varios de los espectáculos más importantes y de gran repercusión.
Nos comenta que para sus espectáculos trabaja principalmente con modelos en dos dimensiones, por lo que la fase inicial de su trabajo se basa en transformar la fachada o superficie a mapear en un modelo bidimensional.
Para este propósito su forma de trabajo consiste en fijar la posición y orientación del proyector en el lugar desde donde se realizará la proyección, para luego capturar las formas y construir el modelo bidimensional.

Una vez modelada la escena, se realiza todo el trabajo creativo y de diseño visual del espectáculo, para lo que utiliza \emph{AfterEffects}, \emph{Photoshop} y \emph{3D Studio}. Luego al momento de la proyección, posiciona nuevamente el proyector en el lugar original, y realiza los ajustes previos. Entre estos ajustes se encuentra la calibración y posicionamiento del proyector en donde fue resaltada la dificultad existente para mover los proyectores que se utilizan en espectáculos de gran porte, los que pueden llegar a pesar hasta 200 kilogramos, lo que dificulta realizar movimientos milimétricos para ajustes finos. Por esto último considera importante la disposición de herramientas de ajuste o calibración del modelo de software sobre la superficie. Destacó que utiliza exactamente el mismo proyector tanto para la obtención de la geometría como para la posterior ejecución del espectáculo, lo cual según su experiencia reduce el trabajo posterior de calibración.

En cuanto al mapeo, fue destacado como muy importante poder tener la posibilidad de modificar en tiempo real el espectáculo. En referencia al tratamiento de las deformaciones que se producen al proyectar sobre una superficie irregular, comentó que maneja diferentes estrategias. Puede tanto utilizar la deformación dada por la superficie junto con deformaciones de la imagen o video a proyectar para lograr un efecto en conjunto, como también intentar modificar el video o imagen para minimizar los efectos dados por la superficie irregular. Comenta que esta última opción es la menos interesante en el ámbito artístico de los \emph{VJs}. Las herramientas que utiliza para la realización de sus espectáculos son \emph{QuickTime}, \emph{Module8}, \emph{VDMX} y \emph{Resolume PC}.

\section{Martin Borini (\emph{VJ Ailaviu})}
La principal inquietud que nos transmitió Martín Borini\cite{Ailaviu} fue en relación al manejo de información tridimensional durante el proceso de producción de un espectáculo y así poder conocer las deformaciones que se darán en la superficie donde se proyectará. También se muestra preocupado por el problema de la correspondencia entre el modelo tridimensional y la superficie, lo que lo lleva a estar interesado en la posibilidad de la calibración de un modelo de ese estilo.
En sus trabajos realizados sobre fachadas ha tomado fotos a nivel desde el mismo lugar donde se coloca el proyector. Según su experiencia esto produce mejores resultados que trabajar sobre planos o medidas tomadas por él. Otro punto en el cual expresó interés fue en casos en los que se utilizan más de un proyector, en donde surge el problema de las costuras o uniones de las imágenes proyectadas por los diferentes proyectores.

\section{Viktor Vicsek}
Viktor Vicsek\cite{Viktorvicsek} es un \emph{VJ} de nivel internacional y se ha establecido contacto con él luego de observar los espectáculos de más relevancia y popularidad referenciados en sitios relacionados al tema. Comenta que ha implementado sus propios módulos de software para asistirle en la producción. También ha realizado una mejora a la técnica de calibración automática de proyectores de Johnny Lee\footnote{http://johnnylee.net/projects/thesis} utilizando cámaras y visión por computadora en lugar de sensores de luz, lo que le ha resultado particularmente útil al trabajar con grandes superficies o fachadas de edificios. También ha implementado su propia línea de tiempo de videos utilizando \emph{Adobe Air}. Para la realización de los mapeos y distorsiones de imágenes y videos utiliza la herramienta de programación visual \emph{VVVV}.

Su proceso de producción consta en tomar fotografías y obtener los planos arquitectónicos de la escena en cuestión y la creación del modelo, según sus necesidades, con herramientas de modelado como \emph{3D Studio}. Posteriormente realiza el mapeo de texturas de imagen o video aplicando efectos que utilizan \emph{shaders} sobre el modelo mediante la utilización de \emph{VVVV}.