\Huge
\textbf{Glosario}

\vspace{10 mm}

\normalsize 

\textbf{Video Jockey}
def. %Video Jockey o VJ se aplica a aquellos creadores que generan sesiones visuales mezclando en directo pistas de video con música u otro tipo de acción.

\textbf{Motor tridimensional}
def.
 
\textbf{Nube de puntos}
def.

\textbf{Malla tridimensional}
def.

\textbf{Visión por computadora}
def.

\textbf{Mapa de bits}
def.

\textbf{Algoritmo de Kruskal}
def.

\textbf{Eje óptico} 
Eje perpendicular al plano imagen que pasa por el centro de la cámara, también llamado centro óptico\cite{OpticalDesign}.

\textbf{Graphic Processing Unit (GPU)}
La unidad de procesamiento gráfico es un procesador suplementario a la unidad de procesamiento central cuya función especifica es el procesamiento de gráficos\cite{GPUWork}.

\textbf{Canal alfa}
The alpha channel is really a mask -- it specifies how the pixel's colors should be merged with another pixel when the two are overlaid, one on top of the other. 

\textbf{Pixel}
Superficie homogénea más pequeña de las que componen una imagen, que se define por su brillo y color\cite{RAE}.

\textbf{OSC}
\emph{Open Sound Control} es un protocolo para comunicación entre computadoras, sintetizadores de sonido y otros dispositivos multimedia que está optimizado para la tecnología de red moderna\cite{OSCProtocol}.

\textbf{UDP}
Protocolo de datagrama de usuario, es un protocolo sin conexión, no confiable, para aplicaciones que no necesitan la asignación de secuencia ni el control de flujo del TCP y que desean utilizar los suyos propios. Este protocolo también se usa ampliamente para consultas de petición y respuesta de una sola ocasión, del tipo cliente-servidor y en aplicaciones en las que la entrega pronta es más importante que la entrega precisa, como las trasmisiones de voz y vídeo\cite{Tanenbaum}.